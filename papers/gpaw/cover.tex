\documentclass[a4paper,10pt]{article}
%\usepackage[margin=2cm]{geometry}


%opening
\title{Hybrid Parallel Programming for Blue Gene/P \\ Cover Letter}
\date{}

\begin{document}
\maketitle

\subsection*{Outline}
In this work we present optimizations of a Grid-based projector-augmented wave method software, GPAW for the Blue Gene/P architecture. The improvements are achieved by using communication techniques suitable for the Blue Gene/P architecture and a programming paradigm which combine shared and distributed memory programming known as hybrid programming.

We evaluate two different hybrid programming approaches. One approach in which inter-node communication is handled individually by every thread and another approach in which one thread handles the inter-node communication on behalf of all the other threads in a node. The work shows that, on the Blue Gene/P, the first approach is clearly superior the latter. 

Different communication techniques are also explained and evaluated. The most significant performance improvement derives from double buffering and communication batching. Since Blue Gene/P is such a well-balanced system it is possible to do a lot of latency hiding with those two techniques.

The work succeeds in scaling a finite-different stencil operation up to 16384 CPU-cores by using double buffering and communication batching. Furthermore we demonstrate a hybrid programming model which is clearly beneficial compared to the original flat programming model. In total an improvement of 1.94 compared to the original implementation is obtained. The results we demonstrate are reasonably general and may be applied to other finite difference codes.

\subsection*{Publication Note}
This paper is an extended version of a paper that we have presented at the ``14th International Workshop on High-Level Parallel Programming Models and Supportive Environment'' with the title ``GPAW optimized for Blue Gene/P using hybrid programming''. The essential research findings are the same as the original paper, but this paper extends the analysis of the findings and places them in a broader context.


\end{document}
