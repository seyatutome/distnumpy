 \documentclass[preprint,3p,times]{elsarticle}
 
 \journal{Scientific International Journal for Parallel and Distributed Computing}

\begin{document}

\begin{frontmatter}

\title{Hybrid Parallel Programming for Blue Gene/P \\ Resubmission notes}

\begin{abstract}
This document is summary of the changes applied to our manuscript "Hybrid Parallel Programming for Blue Gene/P" submitted for publication to SCPE.
\end{abstract}
\end{frontmatter}

We would like to thank for the very useful suggestions given by the three reviewers and the opportunity to resubmit our manuscript.

\section*{Change list}
\begin{itemize}
\item We have fixed the double title in reference number seven results (this fix was motivated by suggestions from reviewer 1).

\item In the result section we have added a reference to the Blue Gene/P section to make it more clear which computing node environment is used for the results (this extension was motivated by suggestions from reviewer 3).

\item Reviewer 1 points out that we need some more motivation for why supercomputers are need. The introduction is therefore extended with a description of why GPAW users find it interesting to have a massively parallel implementation of GPAW that is able to fully utilize a supercomputer.

\item We have extended the introduction with an explanation for why we only focus on the stencil operation and not a whole GPAW simulation (this extension was motivated by suggestions from reviewer 1).

\item We have removed the speculation of the overall performance of a GPAW simulation in further work (this reduction was motivated by suggestions from reviewer 1).

\item The discussion concerning the complexity, at the beginning of subsection 4.1, was a bit too naive and we have therefore extended the section with a more thorough discussion of the computation complexity (this extension was motivated by suggestions from reviewer 2).

\item Reviewer 1 points out that we need some more generalization of our optimizations and we have therefore introduced a general formula for the amount of time used by waiting on non-hidden communication in section 5.1.2. We have made use of the generalization in a new project called DistNumPy\cite{PGAS10}, which has recently been published.

\item We have added a reference to the ``von Neumann bottleneck'' expression in section 3.

\item To make it clear, in the abstract, that hybrid programming is not the only technique used to obtain better performance. We describe that we also make use of blocked communication to improve latency hiding.

\item We have rewritten some parts of section 4.1 to make it clearer why the stencil operation is not embarrassingly parallel.
\end{itemize}


\bibliographystyle{model1b-num-names}
\bibliography{main}
\end{document}